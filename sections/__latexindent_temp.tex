Generative adversarial networks (GANs) is a type of generative model. A GAN consist of two neural networks, a generator network and a discriminator network. The generator outputs data, taking as input a random noise. The discriminator has to classify whether data given as input to it is a real data from the database or a fake one created by the generator. The generator competes against its adversary, the discriminator. As the training for this generative model progresses, the discriminator learns to classify the fake data accurately, while the generator learns to create realistic samples. "An equilibrium is reached when the data created by the generator is indistinguishable from real data". GANs are frequently used in image generation and, they produce sharp images too. A downside for GAN is that it does not have a well-defined loss function, which makes training GANs difficult. Many variations of GANs have been introduced by researchers all around the globe. GANs are one of the most promising approaches to generative modelling present today and extensive research is being done to explore their potential.