\documentclass[12pt, a4paper]{report}

\usepackage{graphicx}
\usepackage{datetime}
\usepackage{setspace}
\usepackage{ragged2e}
\usepackage[left=2cm, right=2cm, top=2cm]{geometry}

\newdateformat{monthyeardate}{\monthname[\THEMONTH], \THEYEAR}

\begin{document}
    \begin{titlepage}
        \newcommand{\HRule}{\rule{\linewidth}{0.5mm}}
        \begin{center}
            \textsc{\LARGE{Generative Adversarial Networks}}\\[0.75cm]
            \textit{A seminar report submitted by} \\[0.25cm]
            \textsc{\Large{Santhisenan A}} \\[0.25cm]
            \textsc{TVE15CS050} \\[0.75cm]
            \textit{to College of Engineering Trivandrum \\[0.25cm] in partial 
            fulfilment of the requirements for the award of the degree}\\[0.75cm]
            \textsc{\Large B.Tech}\\[0.5cm]
            \textit{in}\\[0.5cm]
            \textsc{\LARGE Computer Science and Engineering}\\[0.75cm]
            \includegraphics[width=2.5cm]{images/emblem.pdf}\\[1cm]

            \textsc{\Large{Department of \\[0.25cm] Computer Science and Engineering}}\\
            [0.75cm]
            \textsc{\large{College of Engineering Trivandrum}}\\[0.75cm]
            \monthyeardate\today
        \end{center}
    \end{titlepage}

    % Certificate
    \begin{center}
        \textsc{\Large{Department of Computer Science and Engineering}}\\[0.25cm]
        \textsc{\Large{College of Engineering Trivandrum}}\\[0.75cm]
        \includegraphics[width=2.5cm]{images/emblem.pdf}\\[1cm]
        
        \textsc{\LARGE{Certificate}}\\
        \justify
        % \doublespace
        \begin{doublespace}
            \textit{This is to certify that this report entitled 
            “Generative Adversarial Networks” is a bona fide record of the seminar 
            presented by \textbf{Santhisenan A, Roll No. TVE15CS050} under my 
            guidance towards the partial fulfilment of the requirements for the 
            award of Bachelor of Technology in Computer Science and Engineering of 
            Kerala Technological University.}\\[1cm]
        \end{doublespace}

        \begin{minipage}{0.4\textwidth}
        \begin{flushleft}
        \noindent\textbf{Dr. Ajeesh Ramanujan}\\[0.25cm]
        \textit{Assistant Professor\\[0.25cm]
        Dept. of CSE\\[0.25cm]
        College of Engineering
        Trivandrum}
        \end{flushleft}
        \end{minipage}

    \end{center}

    \newpage
    % Abstract
    \begin{center}
    \section*{Abstract}
    \addcontentsline{toc}{chapter}{Abstract}
    \justify
    \begin{doublespace}
    Generative adversarial nets or GANs can be thought of as a game theoretic 
    approach to deep learning. Every GAN will have two major components, 
    a generator G, and a discriminator D. They are like two adversaries, playing 
    against each other in a game. G tries to generate images that look real from 
    noise, and D tries to identify fake images generated by G. G and D are in a 
    constant battle throughout the training process. G tries to fool D by making 
    realistic fake images and D tries not to get fooled by G. It is important to 
    have good generator, otherwise generator can never fool the discriminator 
    and model may not converge. It is equally important to have a good 
    discriminator, as otherwise images generated by the GAN will be of no use. 
    One of the most important challenges to GAN is that if any one system fails, 
    the whole system fails. A variety of GANs like WGAN, CGAN, InfoGAN, SRGAN 
    etc. has been developed for various applications. GANs is an area in which 
    very active research is being done.
    
    \noindent\textbf{Keywords:} Generative algorithms, Discriminator, Generator
    \end{doublespace}
    \end{center}

    % Table of contents 
    \newpage
    \tableofcontents

    % Chapter 1 Introduction
    \chapter{Introduction}
    \section{Generative Adversarial Networks}
    Generative adversarial networks (GANs) are a class of artificial intelligence 
    algorithms used in unsupervised machine learning, implemented by a system of 
    two neural networks contesting with each other in a zero-sum game framework. 
    They were introduced by Ian Goodfellow et al. in 2014\cite{base}. This technique 
    can generate photographs that look at least superficially authentic to human 
    observers, having many realistic characteristics.\cite{improvedtechniques}

    \section{Generative vs Discriminative algorithms}
    To understand GANs, you should know how generative algorithms work, and for 
    that, contrasting them with discriminative algorithms is instructive. 
    Discriminative algorithms try to classify input data; that is, given the 
    features of a data instance, they predict a label or category to which that 
    data belongs.


    Discriminative algorithms map features to labels. They are concerned solely 
    with that correlation. One way to think about generative algorithms is that 
    they do the opposite. Instead of predicting a label given certain features, 
    they attempt to predict features given a certain label.

    \section{How GANs work}

    One neural network, called the generator, generates new data instances, while 
    the other, the discriminator, evaluates them for authenticity; i.e. the 
    discriminator decides whether each instance of data it reviews belongs to the 
    actual training dataset or not.\cite{fourj}

    Steps taken by a GAN are:
    \begin{itemize}
    \item{The generator takes some random noise and returns an image}
    \item{This generated image is fed to the discriminator along with a stream of 
    images taken the actual dataset}
    \item{The discriminator taken in both real and fake images and returns the 
    probability of the fake image being real.}
    \end{itemize}

    % Bibliography
    \newpage
    \begin{thebibliography}{99}
    \bibitem{base}
        Goodfellow, Ian; Pouget-Abadie, Jean; Mirza, Mehdi; Xu, Bing; Warde-Farley, 
        David; Ozair, Sherjil; Courville, Aaron; Bengio, Joshua 
        \textit{Generative Adversarial Networks 2014}
    
    \bibitem{improvedtechniques}
        Salimans, Tim; Goodfellow, Ian; Zaremba, Wojciech; Cheung, Vicki; 
        Radford, Alec; Chen, Xi. 
        \textit{Improved Techniques for Training GANs 2016}

    \bibitem{fourj}
      \textit{https://deeplearning4j.org/generative-adversarial-network}

\end{thebibliography}
\end{document}
