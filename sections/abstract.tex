\begin{center}
    \section*{Abstract}
    \addcontentsline{toc}{chapter}{Abstract}
    \justify
    \begin{doublespace}
    Generative adversarial nets or GANs can be thought of as a game theoretic 
    approach to deep learning. Every GAN will have two major components, 
    a generator G, and a discriminator D. They are like two adversaries, playing 
    against each other in a game. G tries to generate images that look real from 
    noise, and D tries to identify fake images generated by G. G and D are in a 
    constant battle throughout the training process. G tries to fool D by making 
    realistic fake images and D tries not to get fooled by G. It is important to 
    have good generator, otherwise generator can never fool the discriminator 
    and model may not converge. It is equally important to have a good 
    discriminator, as otherwise images generated by the GAN will be of no use. 
    One of the most important challenges to GAN is that if any one system fails, 
    the whole system fails. A variety of GANs like WGAN, CGAN, InfoGAN, SRGAN 
    etc. has been developed for various applications. GANs is an area in which 
    very active research is being done.
    
    \noindent\textbf{Keywords:} Generative algorithms, Discriminator, Generator
    \end{doublespace}
\end{center}
