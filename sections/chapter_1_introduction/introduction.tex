\singlespacing
\chapter{INTRODUCTION}
In Artificial Intelligence (AI), we can describe machine learning (ML) as one 
of AI's smaller subsets. Machine learning uses statistical techniques to 
computer systems ability to progressively improve its performance on a specific 
task with data without being explicitly programmed.\cite{Samuel59somestudies} 
Unsupervised learning algorithms are a subset of machine learning algorithms 
which tries to describe the structure of unlabelled data. 

Use of generative models\cite{gen_models} is an approach to unsupervised 
learning. The goal of a generative model is to generate data similar to the ones 
in the dataset. 
Generative Adversarial Network (GAN) is a type of Generative Model. Other types 
of generative models include Variational Autoencoders (VAEs) and autoregressive 
models like PixelRNN. GANs have been successfully applied to solve problems in 
various domains like generating images, videos and audio, text to image 
synthesis etc.

GANs were originally introduced by Ian Goodfellow and his collaborators in 
University of Montreal in 2014\cite{gans_basic}.
Yann LeCun, Director of AI Research at Facebook and Professor at NYU called 
adversarial training as {"the most interesting idea in the last 10 years 
in ML"}\cite{yanlecunn_gans}.


