\chapter{GAN: THE IDEA}
\begin{onehalfspace}
    GANs comprises two components, a generator (G) and a discriminator (D). 
    The goal of the generator model is to produce new data similar to the required 
    one. The discriminator's task is to classify the data presented to it as real or 
    fake. "Real" data belong to the original dataset, and "fake" data are those 
    created by the generator.The generative model competes against its adversary, 
    the discriminative model that learns to determine whether a sample is from the 
    model distribution or the data distribution.    
\section{Analogy}
    Generative networks can be thought of as a team of counterfeiters trying 
    to produce fake currency notes and use it without being caught by the 
    police. Here discriminative networks play the role of police trying to 
    detect fake currency. Initially, both the police and counterfeiters are not 
    very experienced, but as the game between them progresses, both parties 
    master what they were doing.  The game continues until the fake currency 
    notes produced by the counterfeiters are indistinguishable from real 
    currency.

\section{A Mathematical Model}

    
    \begin{figure}[h]
        \caption{Generative adversarial networks \cite{gan_image}}
        \centering
        \includegraphics[width=0.8\linewidth]{images/generative-adversarial-network.png}
    \end{figure} 


\end{onehalfspace}

