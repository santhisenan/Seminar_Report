\chapter{INTRODUCTION}
\begin{onehalfspace}
    In Artificial Intelligence (AI), we can think of machine learning (ML) as one 
    of AI's smaller subsets. Machine learning uses statistical techniques to 
    provide computer systems ability to progressively improve its performance on a specific 
    task with data without being explicitly programmed. \cite{Samuel59somestudies} 
    Unsupervised learning algorithms are a subset of machine learning algorithms 
    which tries to describe the structure of unlabelled data. 

    Generative models \cite{openai_genmodels} is an approach to unsupervised 
    learning. The goal of a generative model is to generate data similar to the ones 
    in the dataset. 
    Generative Adversarial Network (GAN) is a type of Generative Model. Other types 
    of generative models include Variational Autoencoders (VAEs) and autoregressive 
    models like PixelRNN. GANs have been successfully applied to solve problems in 
    various domains like generating images, videos and audio, text to image 
    synthesis etc.

    GANs were originally introduced by Ian Goodfellow and his collaborators in 
    University of Montreal in 2014 \cite{gans_basic}.
    Yann LeCun, Director of AI Research at Facebook and Professor at NYU called 
    adversarial training as {"the most interesting idea in the last 10 years 
    in ML"} \cite{yanlecunn_gans}. In this chapter, this report goes through 
    generative classification algorithms.

    \section{Generative Classification Algorithms}
    Consider a classification problem in which we have to classify an input 
    image as that of either a cat \((y = 0)\) or a dog \((y = 1)\).  What an algorithm 
    like logistic regression will try to do is that it will try to find a 
    decision boundary that separates images of cats from that of dogs. Then to 
    classify an image as that of a cat or a dog, the algorithm attempts to find 
    out on which side of the decision boundary does the new image fall.

    We can attempt to solve the above classification problem in a slightly 
    different way. First, going through all the cat images, we can try to build 
    a model of what a cat looks like and then create a separate model for dogs. 
    Later to classify a new animal using this model, we first match the input 
    image against the cat model and then against the dog model. If the picture 
    looks more like the dogs, the classifier outputs \(y = 1\) and \(y = 0\) otherwise.

    Algorithms that try to learn \(p(y|x)\) directly are called discriminative 
    learning algorithms. These algorithms (such as logistic regression) learn 
    the mapping directly from the set of inputs \(X\) to labels \(\{0,1\}\).
    The second category of algorithms that try to model \(p(x|y)\) and \(p(y)\) is 
    called generative learning algorithms. If \(y\) indicates whether an example is 
    a dog \((1)\) or a cat \((0)\),  then \(p(x|y = 1)\) models the distribution of dogs' 
    features and \(p(x|y = 0)\) models the distribution of cats' features.

    In Bayesian statistical inference, a prior probability distribution of an 
    uncertain quantity is the probability distribution that expresses one's 
    beliefs about this quantity before taking evidence into account. Here \(p(y)\) 
    is the class prior.

    After modelling \(p(x|y)\) and \(p(y)\), our algorithm uses the Bayes rule to 
    derive the posterior distribution on \(y\) given \(x\) \cite{ng2002discriminative} :
    \begin{center}
    \(p(y|x) = \frac{p(x|y)p(y)}{p(x)}\).    
    \end{center}
    
    Here, the denominator is given by \(p(x) = p(x|y = 1)p(y = 1) + 
    p(x|y = 0)p(y = 0)\).

    According to OpenAI \cite{openai_genmodels}, generative models are one of the most promising 
    approaches towards understanding the enormous amount of data out there in 
    the real world. Generative models are trained on immense quantities of data 
    from a specific domain to generate data like it. The intuition behind this 
    approach follows a famous quote from Richard Feynman
    \textit{\say{What I cannot create, I do not understand}} \cite{quote_rf}.

    % TODO: add more on generative algorithms like domains in which they are 
    % applied etc.



\end{onehalfspace}


