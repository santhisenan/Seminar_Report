    \section{Generative Adversarial Networks}
    Generative adversarial networks (GANs) are a class of artificial intelligence 
    algorithms used in unsupervised machine learning, implemented by a system of 
    two neural networks contesting with each other in a zero-sum game framework. 
    They were introduced by Ian Goodfellow et al. in 2014\cite{base}. This technique 
    can generate photographs that look at least superficially authentic to human 
    observers, having many realistic characteristics.\cite{improvedtechniques}

    \section{Generative vs Discriminative algorithms}
    To understand GANs, you should know how generative algorithms work, and for 
    that, contrasting them with discriminative algorithms is instructive. 
    Discriminative algorithms try to classify input data; that is, given the 
    features of a data instance, they predict a label or category to which that 
    data belongs.


    Discriminative algorithms map features to labels. They are concerned solely 
    with that correlation. One way to think about generative algorithms is that 
    they do the opposite. Instead of predicting a label given certain features, 
    they attempt to predict features given a certain label.

    \section{How GANs work}

    One neural network, called the generator, generates new data instances, while 
    the other, the discriminator, evaluates them for authenticity; i.e. the 
    discriminator decides whether each instance of data it reviews belongs to the 
    actual training dataset or not.\cite{fourj}

    Steps taken by a GAN are:
    \begin{itemize}
    \item{The generator takes some random noise and returns an image}
    \item{This generated image is fed to the discriminator along with a stream of 
    images taken the actual dataset}
    \item{The discriminator taken in both real and fake images and returns the 
    probability of the fake image being real.}
\end{itemize}